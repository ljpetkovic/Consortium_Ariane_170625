\begin{frame}{Conclusion et perspectives}
	\begin{enumerate}
		\item 	\textit{PatternRank} : la méthode la plus robuste
		\begin{itemize}
			\item capture la sémantique jusqu'aux pentagrammes
			\begin{itemize}
				%				\item quadrigrammes : \textit{sclérose cérébrale tubéreuse hypertrophique} 
				\item \textit{méningite syphilitique hémorragique fibrineuse aiguë}
			\end{itemize}
			\item produit des scores de pertinence plus élevés
			\begin{itemize}
				\item exception : scores BM25 (\textit{SLA}, \textit{embarras parole}) et TF-IDF (\textit{hypnose})
			\end{itemize}
		\end{itemize}
		
		\item 	les termes les plus impactants dans les corpus :
		\begin{itemize}
			\item Charcot : \textit{hystérie}, \textit{astasie-abasie}, \textit{embarras parole}
			\item Autres :  \textit{hypnose}*, \textit{syndrome de Tourette}, \textit{arthropathies tabétiques}
		\end{itemize}

		%		\item Absence des scores pour les termes comme \textit{SEP} et SLA :
		%		\begin{itemize}                                        
			%			\item solution : chercher leurs symptomes ou leurs descriptions :
			%			\begin{itemize}
				%				\item  \textit{amyotrophie spinale progressive}, \textit{secousses nystagmiques}$\dots$
				%			\end{itemize}
			%		\end{itemize}
	\end{enumerate}
	
	\begin{alertblock}{\vspace*{-0.6mm}}
		\centering
		Les résultats sont alignés avec les faits historiques.
	\end{alertblock}
	
	\bigskip
	Recherches futures : tester les \textit{LLM} ou les \textit{LCM} (angl. \textit{Large Concept Models}) ?
\end{frame}